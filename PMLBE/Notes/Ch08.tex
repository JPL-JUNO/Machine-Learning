\chapter{\label{Ch08}}
\section{Demystifying neural networks}
\subsection{Starting with a single-layer neural network}
\subsubsection*{Layers in neural networks}
A layer is a conceptual collection of nodes (also called units), which simulate neurons in a biological brain. The input layer represents the input features, $x$, and each node is a predictive feature, $x$.

The output layer represents the target variable(s). In binary classification, the output layer contains only one node, whose value is the probability of the positive class. In multiclass classification, the output layer consists of $n$ nodes, where $n$ is the number of possible classes and the value of each node is the probability of predicting that class. In regression, the output layer contains only one node, the value of which is the prediction result.
\figures{fig8-1}{A simple shallow neural network}