\chapter{A Tour of Machine Learning Classifiers Using Scikit-Learn\label{Ch03}}
\section{Modeling class probabilities via logistic regression}
\subsection{Logistic regression and conditional probabilities}

Under the logistic model, we assume that there is a linear relationship between the weighted inputs and the log-odds:
\begin{equation}
    logit(p)=w_1x_1+w_2x_2+\cdots+w_mx_m+b=\sum_{i=j}w_jx_j=\textbf{w}^T\textbf{x}+b
\end{equation}

\figures{fig3-3}{Logistic regression compared to Adaline}
\subsection{Learning the model weights via the logistic loss function}
To explain how we can derive the loss function for logistic regression, let’s first define the likelihood, $\mathcal{L}$, that we want to maximize when we build a logistic regression model, assuming that the individual examples in our dataset are independent of one another. The formula is as follows:
\begin{equation}
    \begin{aligned}
        \mathcal{L}(\textbf{w}, b|\textbf{x}) & =p(y|\textbf{x};\textbf{w},b)=\prod_{i=1}^{m}p\left(y^{(i)}|\textbf{x}^{(i)};\textbf{w},b\right) \\
                                              & =\prod_{i=1}^{m}\left(\sigma(z^{(i)})\right)^{y^{(i)}}\left(1-\sigma(z^{(i)})\right)^{1-y^{(i)}}
    \end{aligned}
\end{equation}

In practice, it is easier to maximize the (natural) log of this equation, which is called the \textbf{log-likelihood} function:
\begin{equation}
    \begin{aligned}
        l(\textbf{w}, b|\textbf{x}) & =\log \mathcal{L}(\textbf{w}, b|\textbf{x})                                             \\
                                    & =\sum_{i=1}\left[y^{(i)}\log(\sigma(z^{(i)}))+(1-y^{(i)})\log(1-\sigma(z^{(i)}))\right]
    \end{aligned}
\end{equation}