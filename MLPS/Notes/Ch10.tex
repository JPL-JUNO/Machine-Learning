\chapter{Working with Unlabeled Data – Clustering Analysis\label{Ch01}}
\section{Grouping objects by similarity using k-means}
\subsection{k-means clustering using scikit-learn}
\begin{algorithm}
    \Begin{
        Randomly pick $k$ centroids from the examples as initial cluster centers\;
        \Repeat{
            the cluster assignments do not change or a user-defined tolerance or maximum number of iterations is reached
        }{
            Assign each example to the nearest centroid, $\mu^{(i)}, j \in \{1,\dots, k\}$\;
            Move the centroids to the center of the examples that were assigned to it\;}
    }
    \caption{The k-means algorithm}
\end{algorithm}

A problem with k-means is that one or more clusters can be empty.
\begin{tcolorbox}[title=Feature scaling]
    When we are applying k-means to real-world data using a Euclidean distance metric, we want to make sure that the features are measured on the same scale and apply z-score standardization or min-max scaling if necessary.
\end{tcolorbox}