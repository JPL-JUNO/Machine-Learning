\chapter{用循环神经网络对序列数据建模\label{Ch15}}
\section{用于序列数据建模的循环神经网络}
\subsection{循环神经网络的循环机制}
相邻时刻隐藏层信息的流动时神经网络可以记住过去的信息。通常用回路来表示相邻时刻隐藏层信息的流动。在表示循环神经网络的图中,这个回路也被称为\textbf{循环边}。
\subsection{循环神经网络激活值计算}
单层循环神经网络所有的权重矩阵如下:
\begin{itemize}
    \item $\bm{W}_{xh}$: 输入层 $\bm{x}^{(t)}$ 和隐藏层 $\bm{h}$ 之间的权重矩阵
    \item $\bm{W}_{hh}$: 与循环边相关联的权重矩阵
    \item $\bm{W}_{ho}$: 隐藏层与输出层之间的权重矩阵
\end{itemize}

对于隐藏层,通过输入值的线性组合方式计算净输入 $\bm{z}_h$,即计算两个权重矩阵与对应输入向量的乘法的和,再加上偏置单元 $\bm{b}_h$:
\begin{equation}
    \bm{z}_h^{(t)} = \bm{W}_{xh}\bm{x}^{(t)}+\bm{W}_{hh}\bm{h}^{(t-1)}+\bm{b}_h
\end{equation}
然后,计算隐藏层在 $t$ 时刻的激活值:
\begin{equation}
    \bm{h}^{t}=\sigma_h(\bm{z}_h^{(t)})=\sigma_h(\bm{W}_{xh}\bm{x}^{(t)}+\bm{W}_{hh}\bm{h}^{(t-1)}+\bm{b}_h)
\end{equation}
其中,$\sigma_h(\cdot)$ 为隐藏层的激活函数

如使用组合权重矩阵 $\bm{W}_h=\left[\bm{W}_{xh};\bm{W}_{hh}\right]$,那么公式将变为:
\begin{equation}
    \bm{h}^{t}=\sigma_h\left(\left[\bm{W}_{xh};\bm{W}_{hh}\right]
    \begin{bmatrix}
        \bm{x}^{(t)}   \\
        \bm{h}^{(t-1)} \\
    \end{bmatrix}+\bm{b}_h\right)
\end{equation}

在获得当前时刻隐藏层的激活值后,就可以计算输出层的激活值:
\begin{equation}
    \bm{o}^{(t)}=\sigma_o(\bm{W}_{ho}\bm{h}^{(t)}+\bm{b}_o)
\end{equation}
\figures{fig15-6}{计算单层循环神经网络隐藏层和输出层的激活值}
\subsection{隐藏层循环与输出层循环}
当存在输出层循环连接时,前一时刻输出层的激活值  $\bm{o}^{(t-1)}$ 可以通过以下两种方式循环:
\begin{itemize}
    \item 添加到当前时刻的隐藏层 $\bm{h}^{(t)}$
    \item 添加到当前时刻的输出层 $\bm{o}^{(t)}$
\end{itemize}
\figures{fig15-7}{不同循环连接模型}
\subsection{远距离学习面临的问题}
在实践中,解决序列中远距离依赖问题常用的三种解决方案如下:
\begin{enumerate}
    \item 梯度剪裁
    \item 截断时序方向传播(TBPTT)
    \item 长短期记忆网络
\end{enumerate}
\figures{fig15-8}{计算损失函数梯度时出现的问题}

梯度剪裁是指为梯度设定一个截止值或阈值,如果梯度超过该值则将梯度设为此截止值。而 TBPTT 限制反向传播时刻数量。
\subsection{长短期记忆网络}
\figures{fig15-9}{长短期记忆网络单元的内部结构}
在长短期记忆网络中,前一时刻的单元状态 $\bm{C}^{(t-1)}$ 直接参与当前时刻单元状态 $\bm{C}^{(t)}$ 的计算。记忆单元中的信息流由多个计算单元(通常称为门)控制。在 \autoref{fig15-8} 中,$\bm{x}^{(t)}$ 表示 $t$ 时刻的输入数据,$\bm{h}^{(t-1)}$ 表示 $t-1$ 时刻隐藏层输出。图中有 4 个框,每个框内给出了激活函数(sigmoid 函数 $\sigma$ 或双曲正切函数 $\tanh$、权重和偏置项。在每个框中,需要计算 $\bm{h}^{(t-1)}$ 和 $\bm{x}^{(t)}$ 向量与矩阵的乘积加上 $\bm{b}^{(t)}$,在通过激活函数得到输出值。输出的值经过运算(称为一个门)。$\bigodot$ 代表两个向量对应元素相乘,$\bigoplus$ 代表两个向量元素相加。

长短期基于网络有 3 种不同类型的门,分别为遗忘门、输入门和输出门。

遗忘门 $\bm{f}_t$ 可以重置单元状态,以防状态无限增长。事实上,遗忘门的任务就是决定保留哪部分信息,遗忘哪部分信息。计算公式如下:
\begin{equation}
    \bm{f}_t=\sigma\left(\bm{W}_{xf}\bm{x}^{(t)}+\bm{W}_{hf}\bm{h}^{(t-1)}+\bm{b}_f\right)
\end{equation}

输出门 $\bm{i}_t$ 和候选值 $\tilde{\bm{C}}_t$ 负责更新单元状态,计算公式如下:
\begin{equation}
    \begin{aligned}
        \bm{i}_t         & =\sigma\left(\bm{W}_{xi}\bm{x}^{(t)}+\bm{W}_{hi}\bm{h}^{(t-1)}+\bm{b}_i\right) \\
        \tilde{\bm{C}}_t & =\tanh\left(\bm{W}_{xc}\bm{x}^{(t)}+\bm{W}_{hc}\bm{h}^{(t-1)}+\bm{b}_c\right)  \\
    \end{aligned}
\end{equation}

$t$ 时刻的单元状态计算如下:
\begin{equation}
    \bm{C}^{(t)}=\left(\bm{C}^{(t-1)}\bigodot\bm{f}_t\right)\bigoplus\left(\bm{i}_t\bigodot\tilde{\bm{C}}_t\right)
\end{equation}

输出门 $\bm{o}_t$ 决定如何更新隐藏层的值:
\begin{equation}
    \bm{o}_t=\sigma\left(\bm{W}_{xo}\bm{x}^{(t)}+\bm{W}_{ho}\bm{h}^{(t-1)}+\bm{b}_o\right)
\end{equation}
综上所述,当前时刻隐藏层的输出计算如下:
\begin{equation}
    \bm{h}^{(t)}=\bm{o}_t\bigodot\tanh\left(\bm{C}^{(t)}\right)
\end{equation}