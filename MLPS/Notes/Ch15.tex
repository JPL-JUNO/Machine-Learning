\chapter{用循环神经网络对序列数据建模\label{Ch15}}
\section{用于序列数据建模的循环神经网络}
\subsection{循环神经网络激活值计算}
对于隐藏层,通过输入值的线性组合方式计算净输入 $\bm{z}_h$,即计算两个权重矩阵与对应输入向量的乘法的和,再加上偏置单元 $\bm{b}_h$:
\begin{equation}
    \bm{z}_h^{(t)} = \bm{W}_{xh}\bm{x}^{(t)}+\bm{W}_{hh}\bm{h}^{(t-1)}+\bm{b}_h
\end{equation}
然后,计算隐藏层在 $t$ 时刻的激活值:
\begin{equation}
    \bm{h}^{t}=\sigma_h(\bm{z}_h^{(t)})=\sigma_h(\bm{W}_{xh}\bm{x}^{(t)}+\bm{W}_{hh}\bm{h}^{(t-1)}+\bm{b}_h)
\end{equation}
其中,$\sigma_h(\cdot)$ 为隐藏层的激活函数

如使用组合权重矩阵 $\bm{W}_h=\left[\bm{W}_{xh};\bm{W}_{hh}\right]$,那么公式将变为:
\begin{equation}
    \bm{h}^{t}=\sigma_h\left(\left[\bm{W}_{xh};\bm{W}_{hh}\right]
    \begin{bmatrix}
        \bm{x}^{(t)}   \\
        \bm{h}^{(t-1)} \\
    \end{bmatrix}+\bm{b}_h\right)
\end{equation}

在获得当前时刻隐藏层的激活值后,就可以计算输出层的激活值:
\begin{equation}
    \bm{o}^{(t)}=\sigma_o(\bm{W}_{ho}\bm{h}^{(t)}+\bm{b}_o)
\end{equation}
\figures{fig15-6}{计算单层循环神经网络隐藏层和输出层的激活值}
\subsection{隐藏层循环与输出层循环}
当存在输出层循环连接时,前一时刻输出层的激活值  $\bm{o}^{(t-1)}$ 可以通过以下两种方式循环:
\begin{itemize}
    \item 添加到当前时刻的隐藏层 $\bm{h}^{(t)}$
    \item 添加到当前时刻的输出层 $\bm{o}^{(t)}$
\end{itemize}
\figures{fig15-7}{不同循环连接模型}
\subsection{远距离学习面临的问题}
在实践中,解决序列中远距离依赖问题常用的三种解决方案如下:
\begin{enumerate}
    \item 梯度剪裁
    \item 截断时序方向传播(TBPTT)
    \item 长短期记忆网络
\end{enumerate}
\figures{fig15-8}{计算损失函数梯度时出现的问题}
梯度剪裁是指为梯度设定一个截止值或阈值,如果梯度超过该值则将梯度设为此截止值。而TBPTT限制反向传播时刻数量。