\chapter{Classifying Images with Deep Convolutional Neural Networks\label{Ch14}}
\section{The building blocks of CNNs}
\subsection{Discrete convolutions in one dimension}
\subsubsection*{Determining the size of the convolution output}
The output size of a convolution is determined by the total number of times that we shift the filter, $\bm{w}$, along the input vector. Let's assume that the input vector is of size $n$ and the filter is of size $m$. Then, the size of the output resulting from $\bm{y}=\bm{x}*\bm{w}$ , with padding $p$ and stride $s$, would be determined as follows:
\begin{equation}
    o=\Bigl\lfloor\frac{n+2p-m}{s}\Bigr\rfloor+1
\end{equation}

\subsection{Subsampling layers}
Subsampling is typically applied in two forms of pooling operations in CNNs: max-pooling and
mean-pooling (also known as average-pooling). The pooling layer is usually denoted by $P_{n_1\times n_2}$.

The advantage of pooling is twofold:
\begin{itemize}
    \item Pooling (max-pooling) introduces a local invariance. This means that small changes in a local neighborhood do not change the result of max-pooling. Therefore, it helps with generating features that are more robust to noise in the input data.
    \item Pooling decreases the size of features, which results in higher computational efficiency. Furthermore, reducing the number of features may reduce the degree of overfitting as well.
\end{itemize}

\begin{tcolorbox}[title=Overlapping versus non-overlapping pooling]
    Traditionally, pooling is assumed to be non-overlapping. Pooling is typically performed on non-overlapping neighborhoods, which can be done by setting the stride parameter equal to the pooling size. For example, a non-overlapping pooling layer, $P_{n_1\times n_2}$ , requires a stride parameter $s = (n_1, n_2)$. On the other hand, overlapping pooling occurs if the stride is smaller than the pooling size.
\end{tcolorbox}